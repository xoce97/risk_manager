\documentclass[12pt, a4paper, oneside]{article}
\usepackage[spanish]{babel}
\usepackage[utf8]{inputenc}
\usepackage{graphicx}
\usepackage{hyperref}
\usepackage{listings}
\usepackage{xcolor}
\usepackage{geometry}
\usepackage{fancyhdr}
\usepackage{titlesec}
\usepackage{enumitem}

\geometry{margin=2.5cm}
\setlength{\headheight}{15pt}
\setlength{\parindent}{0pt}
\setlength{\parskip}{1em}

% Configuración de colores para código
\definecolor{codegreen}{rgb}{0,0.6,0}
\definecolor{codegray}{rgb}{0.5,0.5,0.5}
\definecolor{codepurple}{rgb}{0.58,0,0.82}
\definecolor{backcolour}{rgb}{0.95,0.95,0.92}

\lstdefinestyle{mystyle}{
    backgroundcolor=\color{backcolour},
    commentstyle=\color{codegreen},
    keywordstyle=\color{magenta},
    numberstyle=\tiny\color{codegray},
    stringstyle=\color{codepurple},
    basicstyle=\ttfamily\footnotesize,
    breakatwhitespace=false,
    breaklines=true,
    captionpos=b,
    keepspaces=true,
    numbers=left,
    numbersep=5pt,
    showspaces=false,
    showstringspaces=false,
    showtabs=false,
    tabsize=2
}

\lstset{style=mystyle}

% Configuración de headers y footers
\pagestyle{fancy}
\fancyhf{}
\rhead{Sistema de Gestión de Riesgos}
\lhead{Documentación Técnica}
\rfoot{Página \thepage}

\title{
    \vspace{2cm}
    \textbf{Sistema de Gestión de Riesgos} \\
    \Large Documentación Técnica Completa
    \vspace{1cm}
}
\author{
    Departamento de Desarrollo
}
\date{\today}

\begin{document}

\maketitle
\thispagestyle{empty}

\begin{abstract}
Este documento describe el sistema completo de gestión de riesgos desarrollado con FastAPI (backend) y HTML/Bootstrap/JavaScript (frontend). El sistema permite identificar, analizar, evaluar y gestionar riesgos de manera eficiente, con cálculo automático de niveles de riesgo y generación de recomendaciones inteligentes.
\end{abstract}

\newpage
\tableofcontents
\newpage

\section{Introducción}

\subsection{Propósito del Sistema}
El Sistema de Gestión de Riesgos es una aplicación web diseñada para ayudar a organizaciones en la identificación, análisis y gestión de riesgos.

\subsection{Características Principales}
\begin{itemize}
    \item Registro y gestión completa de riesgos
    \item Cálculo automático de niveles de riesgo
    \item Generación de recomendaciones inteligentes
    \item Interfaz web responsive con Bootstrap
    \item API RESTful completa
    \item Base de datos SQLite
\end{itemize}

\subsection{Tecnologías Utilizadas}
\begin{itemize}
    \item Backend: Python 3.8+, FastAPI, SQLAlchemy, Pydantic
    \item Frontend: HTML5, Bootstrap 5, JavaScript, Jinja2
    \item Base de datos: SQLite
\end{itemize}

\section{Arquitectura del Sistema}

\subsection{Estructura del Proyecto}
\begin{verbatim}
risk_manager/
  app/
    main.py
    database.py
    models.py
    schemas.py
    crud.py
    routers/
      risks.py
      categories.py
  templates/
    index.html
    risks.html
    form.html
    risk-detail.html
    categories.html
  static/
    css/style.css
    js/app.js
  requirements.txt
\end{verbatim}

\section{Instalación y Configuración}

\subsection{Requisitos Previos}
\begin{lstlisting}
Python 3.8 o superior
pip (gestor de paquetes de Python)
\end{lstlisting}

\subsection{Instalación de Dependencias}
\begin{lstlisting}[language=bash]
# Crear entorno virtual
python -m venv .venv

# Activar entorno virtual (Windows)
.venv\Scripts\activate

# Instalar dependencias
pip install fastapi uvicorn sqlalchemy
\end{lstlisting}

\subsection{Ejecución del Sistema}
\begin{lstlisting}[language=bash]
uvicorn app.main:app --reload
\end{lstlisting}

URLs de acceso:
\begin{itemize}
    \item Frontend: http://localhost:8000
    \item API Docs: http://localhost:8000/docs
\end{itemize}

\section{Modelo de Datos}

\subsection{Tabla: risks}
\begin{tabular}{|l|l|l|}
\hline
Campo & Tipo & Descripción \\
\hline
id & INTEGER & Identificador único \\
title & VARCHAR & Título del riesgo \\
probability & INTEGER & Probabilidad (1-5) \\
impact & INTEGER & Impacto (1-5) \\
risk\_level & ENUM & Nivel calculado \\
owner & VARCHAR & Responsable \\
category\_id & INTEGER & ID de categoría \\
\hline
\end{tabular}

\subsection{Tabla: risk\_categories}
\begin{tabular}{|l|l|l|}
\hline
Campo & Tipo & Descripción \\
\hline
id & INTEGER & Identificador único \\
name & VARCHAR & Nombre categoría \\
description & VARCHAR & Descripción \\
\hline
\end{tabular}

\section{API Reference}

\subsection{Endpoints de Riesgos}

\subsubsection{GET /risks/}
Obtiene todos los riesgos registrados.

\subsubsection{POST /risks/}
Crea un nuevo riesgo.

\subsubsection{PUT /risks/\{id\}}
Actualiza un riesgo existente.

\subsubsection{DELETE /risks/\{id\}}
Elimina un riesgo.

\subsection{Endpoints de Categorías}

\subsubsection{GET /categories/}
Obtiene todas las categorías.

\subsubsection{POST /categories/}
Crea una nueva categoría.

\section{Frontend y Plantillas}

\subsection{Plantillas HTML}
\begin{itemize}
    \item index.html - Página principal
    \item risks.html - Lista de riesgos
    \item form.html - Formulario de riesgo
    \item risk-detail.html - Detalle con recomendaciones
    \item categories.html - Gestión de categorías
\end{itemize}

\subsection{URLs de Vistas}
\begin{itemize}
    \item http://localhost:8000/ - Dashboard
    \item http://localhost:8000/risks - Lista de riesgos
    \item http://localhost:8000/add-risk - Formulario
    \item http://localhost:8000/risk/1 - Detalle
    \item http://localhost:8000/categories-manager - Categorías
\end{itemize}

\section{Flujo de Trabajo}

\subsection{Registro de Nuevo Riesgo}
\begin{enumerate}
    \item Acceder a /add-risk
    \item Completar formulario
    \item Sistema calcula nivel de riesgo
    \item Se generan recomendaciones
    \item Riesgo se guarda en BD
    \item Redirección a lista de riesgos
\end{enumerate}

\subsection{Proceso de Análisis}
\begin{enumerate}
    \item Entrada: Probabilidad e Impacto
    \item Cálculo: Puntuación = Probabilidad × Impacto
    \item Clasificación:
    \begin{itemize}
        \item 1-4: LOW (Bajo)
        \item 5-10: MEDIUM (Medio)
        \item 11-20: HIGH (Alto)
        \item 21-25: CRITICAL (Crítico)
    \end{itemize}
    \item Salida: Recomendaciones por nivel
\end{enumerate}

\section{Manual de Usuario}

\subsection{Primeros Pasos}
\begin{enumerate}
    \item Crear categorías en /categories-manager
    \item Agregar riesgo en /add-risk
    \item Ver lista en /risks
\end{enumerate}

\subsection{Gestión de Riesgos}
\begin{itemize}
    \item Filtrar por nivel de riesgo
    \item Buscar por título o descripción
    \item Ordenar por probabilidad/impacto
\end{itemize}

\subsection{Acciones Disponibles}
\begin{itemize}
    \item Ver - Detalle completo con recomendaciones
    \item Editar - Modificar información
    \item Eliminar - Remover riesgo
    \item Imprimir - Generar reporte
\end{itemize}

\section{Mantenimiento y Soporte}

\subsection{Backups de Base de Datos}
\begin{lstlisting}[language=bash]
cp risk_management.db risk_management_backup.db
\end{lstlisting}

\subsection{Monitoreo del Sistema}
\begin{itemize}
    \item Verificar logs de la aplicación
    \item Monitorear uso de memoria
    \item Revisar espacio en disco
\end{itemize}

\section{Guía de Resolución de Problemas}

\subsection{Error: Tablas no existentes}
\begin{itemize}
    \item Causa: Tablas no creadas en la base de datos
    \item Solución: Ejecutar creación de tablas
\end{itemize}

\subsection{Error: Configuración CORS}
\begin{itemize}
    \item Causa: Configuración incorrecta de CORS
    \item Solución: Verificar origins en middleware
\end{itemize}

\subsection{Error: Categoría requerida}
\begin{itemize}
    \item Causa: No hay categorías creadas
    \item Solución: Crear categorías primero
\end{itemize}

\section{Plan de Futuro}

\subsection{Mejoras Planeadas}
\begin{itemize}
    \item Sistema de autenticación
    \item Notificaciones por email
    \item Reportes PDF automáticos
    \item API para integraciones
    \item Dashboard con gráficos
\end{itemize}

\end{document}